% Full instructions available at:
% https://github.com/elauksap/focus-beamertheme

\documentclass{beamer}
\usetheme{focus}

\title{Focus: \\ A Minimalist Beamer Theme}
\subtitle{Subtitle}
\author{Author 1 \\ Author 2}
\titlegraphic{\includegraphics[scale=1.25]{focuslogo.pdf}}
\institute{Institute Name \\ Institute Address}
\date{dd mm yyyy}

\begin{document}
    \begin{frame}
        \maketitle
    \end{frame}
    
    \section{Section 1}
    \begin{frame}{Simple frame}
        This is a simple frame.
    \end{frame}

    \begin{frame}[plain]{Plain frame}
        This is a frame with plain style and it is numbered.
    \end{frame}
    
    \begin{frame}[t]
        This frame has an empty title and is aligned to top.
    \end{frame}
    
    \begin{frame}[noframenumbering]{No frame numbering}
        This frame is not numbered and is citing reference \cite{knuth74}.
    \end{frame}
    
    \begin{frame}{Typesetting and Math}
        The packages \texttt{inputenc} and \texttt{FiraSans}\footnote{\url{https://fonts.google.com/specimen/Fira+Sans}}\textsuperscript{,}\footnote{\url{http://mozilla.github.io/Fira/}} are used to properly set the main fonts.
        \vfill
        This theme provides styling commands to typeset \emph{emphasized}, \alert{alerted}, \textbf{bold}, \textcolor{example}{example text}, \dots
        \vfill
        \texttt{FiraSans} also provides support for mathematical symbols:
        \begin{equation*}
            e^{i\pi} + 1 = 0.
        \end{equation*}
    \end{frame}

    \section{Section 2}
    \begin{frame}{Blocks}
        \begin{block}{Block}
            Text.
        \end{block}
        \pause
        \begin{alertblock}{Alert block}
            Alert \alert{text}.
        \end{alertblock}
        \pause
        \begin{exampleblock}{Example block}
            Example \textcolor{example}{text}.
        \end{exampleblock}
    \end{frame}
    
    \begin{frame}{Lists}
        \begin{columns}[t, onlytextwidth]
            \column{0.33\textwidth}
                Items:
                \begin{itemize}
                    \item Item 1
                    \begin{itemize}
                        \item Subitem 1.1
                        \item Subitem 1.2
                    \end{itemize}
                    \item Item 2
                    \item Item 3
                \end{itemize}
            
            \column{0.33\textwidth}
                Enumerations:
                \begin{enumerate}
                    \item First
                    \item Second
                    \begin{enumerate}
                        \item Sub-first
                        \item Sub-second
                    \end{enumerate}
                    \item Third
                \end{enumerate}
            
            \column{0.33\textwidth}
                Descriptions:
                \begin{description}
                    \item[First] Yes.
                    \item[Second] No.
                \end{description}
        \end{columns}
    \end{frame}

    \begin{frame}[focus]
        Thanks for using \textbf{Focus}!
    \end{frame}
    
    \appendix
    \begin{frame}{References}
        \nocite{*}
        \bibliography{demo_bibliography}
        \bibliographystyle{plain}
    \end{frame}
    
    \begin{frame}{Backup frame}
        \usebeamercolor[fg]{normal text}
        This is a backup frame, useful to include additional material for questions from the audience.
        \vfill
        The package \texttt{appendixnumberbeamer} is used not to number appendix frames.
    \end{frame}
\end{document}
